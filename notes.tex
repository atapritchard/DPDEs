\documentclass[10pt]{article}


\usepackage{amssymb,amsthm,amsmath}
\usepackage{enumerate}
\usepackage{graphicx,color}
\usepackage[hidelinks]{hyperref}
%\usepackage{refcheck}

\newcommand{\dd}{\mathrm{d}}
\newcommand{\E}{\mathbb{E}}
\newcommand{\1}{\textbf{1}}
\newcommand{\R}{\mathbb{R}}
\newcommand{\C}{\mathbb{C}} 
\newcommand{\Z}{\mathbb{Z}}
\newcommand{\N}{\mathbb{N}}
\newcommand{\p}[1]{\mathbb{P}\left( #1 \right)}
\newcommand{\scal}[2]{\left\langle #1, #2 \right\rangle}
\newcommand{\red}{\color{red}}
\newcommand{\shift}{\vdash}
\newcommand{\lL}{\mathcal{L}}

\DeclareMathOperator{\Var}{Var}
\DeclareMathOperator{\sgn}{sgn}

\usepackage[paper=a4paper, left=1.3in, right=1.3in, top=1in, bottom=1in]{geometry}
\linespread{1.3}
\pagestyle{plain}

\newtheorem{theorem}{Theorem}
\newtheorem{lemma}[theorem]{Lemma}
\newtheorem{corollary}[theorem]{Corollary}

\theoremstyle{remark}
\newtheorem{remark}[theorem]{Remark}


\newtheorem{conjecture}{Conjecture}

\theoremstyle{definition}
\newtheorem{prop}{Prop}
\newtheorem{lemma}{Lemma}
\newtheorem{remark}{Remark}
\newtheorem{defi}{Def}
\newtheorem{apps}{Application}
\newtheorem{quest}{Question}
\newtheorem{ans}{Answer}
\newtheorem{interest}{Interesting}
\newtheorem{theme}{Theme}
\newtheorem{theorem}{Theorem}
\newtheorem{example}{Example}



\begin{document}

\section{Notes on Deep Partial Differential Equations}

\subsection{The Deep Galerkin Method}

Drawbacks of grid methods

Numerical methods based on grids often fail when dimensionality becomes too large. (b/c grid points exponential in dimension). Stabiliy also much huge mesh sizes

Advantages of DGM

Mesh free. 

?Computational graph? 

Training Method:

Randomly sample points from region where function defined

?Why do expect having a good fit on the boundary to be a good fit inside?
	If harmonic then this makes sense. Maybe work for good class of PDEs?

Mathematical Formulation:

We define $f(t,x;\theta)$ with parameter $\theta$(fitting parameter)  - actually these are given by neural network

?How do we compute differential operator loss?
-Because we have f approximation defiend everywhere

$\nu$ term is density, describes mass of space. Distribution of space

Approximation works for class of quasilinear parabolic PDE
2018 Sirignano and Spiliopoulos *should look into this*

Implmentation Details:

DGM layers: input - hidden ... hidden - output
	Takes as input original x and input from previous layer

?Highway network?

Drawbacks:

**Number of parameters in each hidden layer of DGm is roughly eight times bigger than same number in usual dense layer

Second orders derivatives quadratically costly: Could use backprop but instead use finite differenc scheme

?Numerous choices made outlined in 47 toronto?

\section{Model Notes}

Keep in mind our approximations differ from closed form cause we assume trapped in compact box(as opposed to analytical solution)

Our perfomance is so variable?

Upshot of our model is that our output is smooth whereas other outputs continuous.


\subsection{Comparisons}

runge ketta?

\subsection{Ideas}

Approximate other functions? Ray tracing equation?

\subsection{Questions}

Where does the parallelism come in?

\section{Heat Equation}

\begin{verbatim}
	https://en.wikipedia.org/wiki/Heat_equation
\end{verbatim}

In most general case not always possible to get closed form solution. 

Need fundamental solutions, not always even closed form.


\end{document}



